\documentclass[a4paper,12pt]{article}

% Pacotes úteis
\usepackage[utf8]{inputenc}       % Codificação UTF-8
\usepackage[T1]{fontenc}          % Suporte para acentuação
\usepackage{lmodern}              % Fonte moderna
\usepackage{graphicx}             % Para inclusão de imagens
\usepackage{amsmath}              % Pacote de matemática
\usepackage{amssymb}              % Símbolos matemáticos
\usepackage{geometry}             % Controle das margens
\usepackage{setspace}             % Controle de espaçamento
\usepackage{indentfirst}          % Indentar primeiro parágrafo de cada seção
\usepackage{hyperref}             % Links clicáveis
\usepackage{float}                % Controle de posição de elementos flutuantes (figuras, tabelas)
\usepackage{titlesec}             % Personalização de títulos
         % Para imagens (caso use)
\usepackage{xcolor}           % Para cores nos textos (opcional)
\usepackage{listings}         % Para código C++ (se for incluir trechos)




\documentclass{article}
\usepackage{minted}

% Configurações opcionais para personalizar
\setminted{
    breaklines=true,      % Quebra automaticamente linhas longas
    breakanywhere=true,   % Permite quebrar palavras (opcional)
    frame=lines,          % Adiciona linhas ao redor do código
    fontsize=\small,      % Tamanho da fonte
    breaklines=true,
    breaksymbol=,         % Adiciona um símbolo (opcional)
    breakindent=10pt       % Indenta as linhas quebradas
}



% Configurações de página
% \geometry{left=3cm, right=2cm, top=2cm, bottom=2cm}
% \onehalfspacing                      % Espaçamento de 1,5 entre linhas

% % Cabeçalho com título e autores
% \title{Avaliação de LLM's na extração de dados médicos de notas clínicas}
% \author{
%     Ana Júlia Amaro Pereira Rocha \\ 
%     Maria Eduarda Mesquita Magalhães\\ 
%     Mariana Fernandes Rocha \\ 
%     Paula Eduarda de Lima \\
%     \\ Orientador: Walter Sande
% }
% \date{\today}                         % Data do relatório

% \begin{document}

% \documentclass[12pt,a4paper]{report}
\usepackage{graphicx}
\usepackage{titling}

\documentclass[a4paper,12pt]{article}

\usepackage[portuguese]{babel} % Ativa o português para hifenização
\usepackage[utf8]{inputenc}    % Suporte a caracteres acentuados
\usepackage[T1]{fontenc}       % Fonte com suporte a caracteres especiais
\usepackage{xcolor}
\usepackage[ colorlinks=true, linkcolor=blue,     urlcolor=blue,  citecolor=blue ]{hyperref}
\begin{document}

\begin{titlepage}
    \begin{center}

        \vspace{1cm}
        \begin{minipage}{0.45\textwidth}
            \centering
            \includegraphics[width=1.2\textwidth]{logo_fgv.png}    
        \end{minipage}
        \vspace{2cm}

        \rule{1\textwidth}{0.4pt} \\ % Linha horizontal personalizada
        \vspace{0.3cm}
        {\Huge \textbf{Plataforma de Monitoramento Epidemiológico em Tempo Real e Histórico}} \\
        \vspace{0.2cm}
        \vspace{0.5cm}\\
        {\Large \textbf{A2 Computação Escalável}}\\
        \rule{1\textwidth}{0.4pt} % Linha horizontal personalizada


        \vspace{0.5cm}
        {\Large \textbf{FGV EMAp}} \\
        \vspace{2cm}
        
        

        
        
        % % Unidade e curso
        % {\Large \textbf{FGV EMAp}}\\[2cm]
        
        % Autores
        {\large 
            \textbf{Ana Júlia Amaro Pereira Rocha} \\ 
            \textbf{Henrique Borges Carvalho} \\
            \textbf{Maria Eduarda Mesquita Magalhães}\\
            \textbf{Mariana Fernandes Rocha} \\
            \textbf{Paula Eduarda de Lima}}\\[1.5cm]
        
        % Informações adicionais
        {\large 
            Ciência de Dados e Inteligência Artificial \\ 
            5º Período}\\[2cm]
        
         % Data
        \vfill
        {\large Rio de Janeiro, 2025}

        
    \end{center}
\end{titlepage}

\tableofcontents
\newpage


\section{Introdução}

\section{Modelagem}


\section{Mock}

As fontes de dados serão simulações de hospitais,  Secretaria de Saúde (SS) e Organização Mundial da Saúde (OMS). Receberemos conjuntos de dados advindos dessas fontes semanalmente. Além disso, temos dois tipos de segmentação regional: Ilhas, mais geral, e regiões, que são segmentações das ilhas, mais especifíco. A simulação apresenta 20 ilhas, cada uma identificada por um CEP de 2 dígitos, cada ilha tem 5 regiões, com CEPs de 5 dígitos, sendo os dois primeiros dígitos o CEP da ilha (ex: ilha 12 $\rightarrow$ regiões 12001 a 12005).


O objeto Mock é feito em python e implementa a criação de três tipos de tabelas diferes: SQLite, csv e txt.
\vspace{0.5cm}

\textbf{Fontes e dados gerados}

\begin{itemize}
    \item \textcolor{purple}{\textbf{OMS (\texttt{oms\_mock.txt})}}\\
    Dados agregados por ilha (CEP de 2 dígitos).\\
    Inclui número de óbitos, população, recuperados, vacinados e data.\\
    Gerado em formato \texttt{.txt} com tabulações.
    
    \vspace{0.5em}
    
    \item \textcolor{red}{\textbf{Hospitais (\texttt{hospital\_mock\_*.csv})}}\\
    Dados individuais por paciente, com CEPs de 5 dígitos (regiões).\\
    Contém informações como internação, idade, sexo, sintomas e data.\\
    Gera múltiplos arquivos \texttt{.csv}, simulando diferentes hospitais.
    
    \vspace{0.5em}
    
    \item \textcolor{orange}{\textbf{Secretaria de Saúde (\texttt{secretary\_data.db})}}\\
    Banco de dados SQLite com tabela \texttt{pacientes}.\\
    Dados por região (CEP de 5 dígitos): diagnóstico, vacinação, escolaridade, população e data.\\
    Registros inseridos diretamente em um banco relacional.
\end{itemize}


\section{Tratadores}


\section{Dashboard}


\subsection*{Exemplos na prática}

% mostra imagens exemplo

\section{Pipeline}

\subsubsection*{Pontos fortes da abordagem}


\begin{itemize}
    \item 
\end{itemize}

\subsubsection*{Pontos fracos da abordagem}

\begin{itemize}
    \item 
\end{itemize}

\subsection*{Tratamento}
\subsubsection*{Pontos fortes específicos da abordagem}

\subsubsection*{Pontos fracos específicos da abordagem}


\section{Análise de tempo de execução}

\begin{figure}[H]
\end{figure}


\end{document}