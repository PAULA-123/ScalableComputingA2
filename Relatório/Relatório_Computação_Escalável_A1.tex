\documentclass[a4paper,12pt]{article}

% Pacotes úteis
\usepackage[utf8]{inputenc}       % Codificação UTF-8
\usepackage[T1]{fontenc}          % Suporte para acentuação
\usepackage{lmodern}              % Fonte moderna
\usepackage{graphicx}             % Para inclusão de imagens
\usepackage{amsmath}              % Pacote de matemática
\usepackage{amssymb}              % Símbolos matemáticos
\usepackage{geometry}             % Controle das margens
\usepackage{setspace}             % Controle de espaçamento
\usepackage{indentfirst}          % Indentar primeiro parágrafo de cada seção
\usepackage{hyperref}             % Links clicáveis
\usepackage{float}                % Controle de posição de elementos flutuantes (figuras, tabelas)
\usepackage{titlesec}             % Personalização de títulos
         % Para imagens (caso use)
\usepackage{xcolor}           % Para cores nos textos (opcional)
\usepackage{listings}         % Para código C++ (se for incluir trechos)




\documentclass{article}
\usepackage{minted}

% Configurações opcionais para personalizar
\setminted{
    breaklines=true,      % Quebra automaticamente linhas longas
    breakanywhere=true,   % Permite quebrar palavras (opcional)
    frame=lines,          % Adiciona linhas ao redor do código
    fontsize=\small,      % Tamanho da fonte
    breaklines=true,
    breaksymbol=,         % Adiciona um símbolo (opcional)
    breakindent=10pt       % Indenta as linhas quebradas
}



% Configurações de página
% \geometry{left=3cm, right=2cm, top=2cm, bottom=2cm}
% \onehalfspacing                      % Espaçamento de 1,5 entre linhas

% % Cabeçalho com título e autores
% \title{Avaliação de LLM's na extração de dados médicos de notas clínicas}
% \author{
%     Ana Júlia Amaro Pereira Rocha \\ 
%     Maria Eduarda Mesquita Magalhães\\ 
%     Mariana Fernandes Rocha \\ 
%     Paula Eduarda de Lima \\
%     \\ Orientador: Walter Sande
% }
% \date{\today}                         % Data do relatório

% \begin{document}

% \documentclass[12pt,a4paper]{report}
\usepackage{graphicx}
\usepackage{titling}

\documentclass[a4paper,12pt]{article}

\usepackage[portuguese]{babel} % Ativa o português para hifenização
\usepackage[utf8]{inputenc}    % Suporte a caracteres acentuados
\usepackage[T1]{fontenc}       % Fonte com suporte a caracteres especiais
\usepackage{xcolor}
\usepackage[ colorlinks=true, linkcolor=blue,     urlcolor=blue,  citecolor=blue ]{hyperref}
\begin{document}

\begin{titlepage}
    \begin{center}

        \vspace{1cm}
        \begin{minipage}{0.45\textwidth}
            \centering
            \includegraphics[width=1.2\textwidth]{logo_fgv.png}    
        \end{minipage}
        \vspace{2cm}

        \rule{1\textwidth}{0.4pt} \\ % Linha horizontal personalizada
        \vspace{0.3cm}
        {\Huge \textbf{Pipeline escalável para processamento de dados}} \\
        \vspace{0.2cm}
        \vspace{0.5cm}\\
        {\Large \textbf{A2 Computação Escalável}}\\
        \rule{1\textwidth}{0.4pt} % Linha horizontal personalizada


        \vspace{0.5cm}
        {\Large \textbf{FGV EMAp}} \\
        \vspace{2cm}
        
        

        
        
        % % Unidade e curso
        % {\Large \textbf{FGV EMAp}}\\[2cm]
        
        % Autores
        {\large 
            \textbf{Ana Júlia Amaro Pereira Rocha} \\ 
            \textbf{Henrique Borges Carvalho} \\
            \textbf{Maria Eduarda Mesquita Magalhães}\\
            \textbf{Mariana Fernandes Rocha} \\
            \textbf{Paula Eduarda de Lima}}\\[1.5cm]
        
        % Informações adicionais
        {\large 
            Ciência de Dados e Inteligência Artificial \\ 
            5º Período}\\[2cm]
        
         % Data
        \vfill
        {\large Rio de Janeiro, 2025}

        
    \end{center}
\end{titlepage}

\tableofcontents
\newpage


\section{Introdução}

Este relatório detalha o desenvolvimento de um pipeline escalável para processamento de dados,
conforme proposto no Trabalho 2 da disciplina de Computação Escalável. O objetivo principal é 
evoluir o projeto desenvolvido no Trabalho 1, transformando-o em uma solução distribuída 
capaz de aumentar significativamente sua capacidade de escala. Para atingir este fim, a arquitetura 
da solução foi redefinida, aplicando padrões arquiteturais e tecnologias apresentadas em aula, visando eficiência e escalabilidade.

A solução será implantada em um ambiente de nuvem, especificamente na Amazon Web Services (AWS),
embora seja projetada para também ser executada em uma única máquina para fins de desenvolvimento. 
Este documento abordará a modelagem da arquitetura e do banco de dados, as principais
decisões de projeto, e a apresentação e discussão dos resultados experimentais obtidos.

\section{Modelagem}

Este projeto implementa uma pipeline de processamento de dados distribuída e modular, baseada em \textbf{Apache Kafka}, \textbf{Apache Spark} (nos tratadores), e containers \textbf{Docker} para orquestração dos serviços. Quando executado localmente, todo o ambiente é gerenciado por Docker, garantindo isolamento, reprodutibilidade e facilidade de implantação.

\subsection{Arquitetura Geral}

O ambiente é definido via \texttt{docker-compose.yml}, que orquestra múltiplos containers divididos em três categorias principais:

\begin{itemize}
  \item \textbf{Serviços de infraestrutura e mensageria}: 
          \begin{itemize}
          \item \textbf{Zookeeper}: Gerencia a coordenação dos brokers Kafka.
          \item \textbf{Kafka}: Middleware de mensageria baseado em tópicos.
          \end{itemize}
  \item \textbf{Geradores e tratadores de dados}:
  \begin{itemize}
  \item \textbf{mock-generator}: Gera dados sintéticos e os envia ao Kafka.
  \item \textbf{tratador-limpeza}: Realiza a limpeza inicial dos dados.
  \item \textbf{tratador-filtragem}: Aplica filtros aos dados limpos e publica em novos tópicos.
  \item \textbf{tratador-media}: Focado em estatísticas específicas, como médias por região.
  \item \textbf{tratador-merge}: Responsável por unificar dados vindos de diferentes tratadores em um único artefato de saída.
    \item \textbf{tratador-agrupar-colunas}: Agrupar variáveis relacionadas e criar estruturas auxiliares para análise estatística posterior.
    \item \textbf{tratador-correlacao}: Calcular correlação entre variáveis (ex: vacinação vs. número de casos).
    \item \textbf{tratador-desvioPadrao}: Avaliar a variabilidade dos dados, detectar instabilidades e anomalias.
    \item \textbf{tratador-mediamovel}: Aplicar suavização dos dados no tempo, útil para observar tendências (ex: 7 dias, 14 dias).
    \item \textbf{tratador-regressao}: Aplicar modelos estatísticos para prever variáveis (ex: número de casos futuros).
\end{itemize}
\end{itemize}

\subsection{Resumo}

\begin{itemize}
  \item A arquitetura segue o padrão \textbf{event-driven}, com os tratadores consumindo e publicando dados via Kafka.
  \item O uso de \textbf{volumes persistentes} permite manter os dados entre execuções e facilita o rastreamento de artefatos.
  \item A divisão modular permite escalar tratadores de forma independente.
  \item \textbf{O ambiente local é executado inteiramente em containers Docker}, permitindo desenvolvimento e testes isolados.
\end{itemize}

\section{Mock}

As fontes de dados serão simulações de hospitais,  Secretaria de Saúde (SS) e Organização Mundial da Saúde (OMS). Receberemos conjuntos de dados advindos dessas fontes semanalmente. Além disso, temos dois tipos de segmentação regional: Ilhas, mais geral, e regiões, que são segmentações das ilhas, mais especifíco. A simulação apresenta 20 ilhas, cada uma identificada por um CEP de 2 dígitos, cada ilha tem 5 regiões, com CEPs de 5 dígitos, sendo os dois primeiros dígitos o CEP da ilha (ex: ilha 12 $\rightarrow$ regiões 12001 a 12005).

\vspace{0.5cm}

\textbf{Fontes e dados gerados}

\begin{itemize}
    \item \textcolor{purple}{\textbf{OMS}}\\
    Dados agregados por ilha (CEP de 2 dígitos).\\
    Inclui número de óbitos, população, recuperados, vacinados e data.
    
    \vspace{0.5em}
    
    \item \textcolor{red}{\textbf{Hospitais }}\\
    Dados individuais por paciente, com CEPs de 5 dígitos (regiões).\\
    Contém informações como internação, idade, sexo, sintomas e data.
    
    \vspace{0.5em}
    
    \item \textcolor{orange}{\textbf{Secretaria de Saúde}}\\
    Dados por região (CEP de 5 dígitos): diagnóstico, vacinação, escolaridade, população e data..
\end{itemize}


\section{Tratadores}

\section{Dashboard}

\subsection*{Exemplos na prática}

\section{Pipeline}

\subsubsection*{Pontos fortes da abordagem}

\subsubsection*{Pontos fracos da abordagem}

\subsection*{Tratamento}

\subsubsection*{Pontos fortes específicos da abordagem}

\subsubsection*{Pontos fracos específicos da abordagem}

\section{Análise de tempo de execução}



\end{document}